\documentclass[12pt,a4paper,sans]{moderncv}        % possible options include font size ('10pt', '11pt' and '12pt'), paper size ('a4paper', 'letterpaper', 'a5paper', 'legalpaper', 'executivepaper' and 'landscape') and font family ('sans' and 'roman')

% moderncv themes
\moderncvstyle{classic}                             % style options are 'casual' (default), 'classic', 'banking', 'oldstyle' and 'fancy'
\moderncvcolor{black}                               % color options 'black', 'blue' (default), 'burgundy', 'green', 'grey', 'orange', 'purple' and 'red'
\renewcommand{\sfdefault}{lmss}         % to set the default font; use '\sfdefault' for the default sans serif font, '\rmdefault' for the default roman one, or any tex font name
\renewcommand{\rmdefault}{ppl}         % to set the default font; use '\sfdefault' for the default sans serif font, '\rmdefault' for the default roman one, or any tex font name
%\nopagenumbers{}                                  % uncomment to suppress automatic page numbering for CVs longer than one page

% Allows a command to be repeated.
\usepackage{expl3}
\ExplSyntaxOn
\cs_new_eq:NN \Repeat \prg_replicate:nn
\ExplSyntaxOff

\usepackage[scale=0.80]{geometry}

\usepackage{tikz}

% personal data
\name{David}{Vaillant}
%\title{Resumé title}                               % optional, remove / comment the line if not wanted
\address{San Jose, CA}{}{}% optional, remove / comment the line if not wanted; the "postcode city" and "country" arguments can be omitted or provided empty
\phone[mobile]{(408)~439~7551}                   % optional, remove / comment the line if not wanted; the optional "type" of the phone can be "mobile" (default), "fixed" or "fax"
\email{dvaillant@gmail.com}                               % optional, remove / comment the line if not wanted
%\homepage{vaillant.io}                         % optional, remove / comment the line if not wanted
\social[github]{D-Vaillant}                              % optional, remove / comment the line if not wanted

\begin{document}
%-----       letter       ---------------------------------------------------------
% recipient data
\recipient{Company}{Re: Position}
\date{DD mon YYYY}
\opening{To whom it may concern,}
\closing{Best regards,}
%\enclosure[Attached]{curriculum vit\ae{}}          % use an optional argument to use a string other than "Enclosure", or redefine \enclname
\makelettertitle

Firstly, thank you for taking the time to read this cover letter. I have the suspicion that, in most cases, the mere act of submitting a cover letter is sufficient and the actual content of the letter is besides the point. But if you've read this far then I can only assume that you intend on reading this whole thing from start to finish.

There are a few things I would like to convey here that my resume cannot.

I would describe one of my skills as \textit{common reason}, a faculty that is common to programming, mathematics, data analysis, and philosophical thought. My schooling was in abstract mathematics and philosophy and it was not particularly difficult to pivot to the broad subject of "computers". After college I installed Debian on a laptop, taught myself Python, and worked through enough projects to get to the point where I felt extremely comfortable with the language. [TK - I hate this. It's so pretentious.] There are plenty of overlaps - computers are made of boolean algebra; the classical logic of the philosopher Frege is found in almost every programming language. But this is all just to say that my educational background is relevant here.

Following from the above, I enjoy novelty. To fix a recovered television, I purchased a soldering kit and taught myself how to replace some blown capacitors. When a router of mine seemed too hot to the touch, I used a Bluetooth-based temperature sensor to verify this, installed a fan system, and then verified that it worked. I once taught my cousins how to count to 35 using their hands (and when they get older, 2047). I once started a now-defunct streetwear brand between jobs. This cover letter is written using Latex and compiled via Make, to automatically delete the artifacts of the PDF generation. In short, I'm excited to try to solve new problems and try new things.

I've been told that I can be funny and, in most cases, this is meant as a compliment. I subscribe to "Double Simulationism", the theory that the universe is a simulation. and that that simulation is also in a simulation (I strongly reject Triple Simulationism but am sympathetic to Quadruple Simulationism). Given the choice between doing something for an 3 hours every month and spending 15 hours automating the work away I will happily choose the latter, if not just for me then for the people who will follow. [TK - God, is this too morally? I guess that's not bad.] And I genuinely would like to be able to help people in some meaningful way, even if it's just the people directly around it.

[TK - At this point I can write something that's specific to the job posting.] At the very least I would be interested to learn more about X and [TK - question that's specific to X]. 

\makeletterclosing

%\clearpage\end{CJK*}                              % if you are typesetting your resume in Chinese using CJK; the \clearpage is required for fancyhdr to work correctly with CJK, though it kills the page numbering by making \lastpage undefined
\end{document}

