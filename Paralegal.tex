%% start of file `template.tex'.
%% Copyright 2006-2015 Xavier Danaux (xdanaux@gmail.com).
%
% This work may be distributed and/or modified under the
% conditions of the LaTeX Project Public License version 1.3c,
% available at http://www.latex-project.org/lppl/.

\documentclass[12pt,a4paper,sans]{moderncv}        % possible options include font size ('10pt', '11pt' and '12pt'), paper size ('a4paper', 'letterpaper', 'a5paper', 'legalpaper', 'executivepaper' and 'landscape') and font family ('sans' and 'roman')

% moderncv themes
% style options are 'casual' (default), 'classic', 'banking', 'oldstyle' and 'fancy'
\moderncvstyle{banking}
% color options 'black', 'blue' (default), 'burgundy', 'green', 'grey', 'orange', 'purple' and 'red'
\moderncvcolor{burgundy}
\renewcommand{\sfdefault}{lmss}         % to set the default font; use '\sfdefault' for the default sans serif font, '\rmdefault' for the default roman one, or any tex font name
\renewcommand{\rmdefault}{ppl}         % to set the default font; use '\sfdefault' for the default sans serif font, '\rmdefault' for the default roman one, or any tex font name
% uncomment to suppress automatic page numbering for CVs longer than one page
\nopagenumbers{}

% Allows a command to be repeated.
\usepackage{expl3}
\ExplSyntaxOn
\cs_new_eq:NN \Repeat \prg_replicate:nn
\ExplSyntaxOff

% Put at the end!
\usepackage{anyfontsize}
\newenvironment{hidden}{\color{white}\fontsize{1}{1}\selectfont}{}
% Uncomment if we're not using `anyfontsize`. Less ideal, but....
% Maybe adding in a "subscript" thing can shrink it even more?
%\newenvironment{hidden}{\color{white}\tiny}{}

% character encoding
%\usepackage[utf8]{inputenc}                       % if you are not using xelatex ou lualatex, replace by the encoding you are using
%\usepackage{CJKutf8}                              % if you need to use CJK to typeset your resume in Chinese, Japanese or Korean

% adjust the page margins
\usepackage[scale=0.80]{geometry}
%\setlength{\hintscolumnwidth}{3cm}                % if you want to change the width of the column with the dates
%\setlength{\makecvtitlenamewidth}{10cm}           % for the 'classic' style, if you want to force the width allocated to your name and avoid line breaks. be careful though, the length is normally calculated to avoid any overlap with your personal info; use this at your own typographical risks...

\usepackage{tikz}

% personal data
\name{David}{Vaillant}
%\title{Resumé title}                               % optional, remove / comment the line if not wanted
\address{San Jose, CA}{}{}% optional, remove / comment the line if not wanted; the "postcode city" and "country" arguments can be omitted or provided empty
\phone[mobile]{(408)~439~7551}                   % optional, remove / comment the line if not wanted; the optional "type" of the phone can be "mobile" (default), "fixed" or "fax"
\email{david@vaillant.io}                               % optional, remove / comment the line if not wanted
%\homepage{vaillant.io}                         % optional, remove / comment the line if not wanted
%\social[linkedin]{john.doe}                        % optional, remove / comment the line if not wanted
%\social[twitter]{jdoe}                             % optional, remove / comment the line if not wanted
%\extrainfo{}
%\photo[64pt][0.4pt]{picture}                       % optional, remove / comment the line if not wanted; '64pt' is the height the picture must be resized to, 0.4pt is the thickness of the frame around it (put it to 0pt for no frame) and 'picture' is the name of the picture file
%\quote{}

% bibliography adjustements (only useful if you make citations in your resume, or print a list of publications using BibTeX)
%   to show numerical labels in the bibliography (default is to show no labels)
%\makeatletter\renewcommand*{\bibliographyitemlabel}{\@biblabel{\arabic{enumiv}}}\makeatother
%   to redefine the bibliography heading string ("Publications")
%\renewcommand{\refname}{Articles}

% bibliography with mutiple entries
%\usepackage{multibib}
%\newcites{book,misc}{{Books},{Others}}

%\newcommand{\tikzcircle}[2][color2,fill=color1]{\tikz[baseline=-0.5ex]\draw[#1,radius=#2] (0,0) circle ;}%
%\newcommand{\progress}[1]{\Repeat{#1}{\tikzcircle{2.75pt}\,}\Repeat{5-#1}{\tikzcircle[color2,fill=white]{2.75pt}\,}}%
%----------------------------------------------------------------------------------
%            content
%----------------------------------------------------------------------------------
\begin{document}
%-----       resume       ---------------------------------------------------------
\makecvtitle

\section{Education}
	\cventry{2010--2014}{BSc. Mathematics}{University of California, Santa Barbara}{}{}{Studied combinatorics, graph theory, linear algebra, and the foundations of statistics.}  % arguments 3 to 6 can be left empty

	\cventry{}{BA. Philosophy}{University of California, Santa Barbara}{}{}{Specialization in logic and the philosophy of mathematics.}  % arguments 3 to 6 can be left empty
%\cventry{year--year}{Degree}{Institution}{City}{\textit{Grade}}{Description}

	\cventry{2017--Present}{Paralegal Certification}{West Valley Community College}{}{}{}

\section{Experience}
\cventry{2015--2016}{Technical Consultant}{Desert Farms Camel Milk}{Santa Monica}{}{%
\begin{itemize}%
\item Used the Shopify API to query customer orders and programmatically organized those into dynamically generated spreadsheets, saving hours of manual labor.
\item Improved the performance and responsivity for the online storefront by correcting a crucial underlying software defect, leading to increased customer satisfaction and retention
\item Played a key role in requisitioning vital hardware which ultimately eliminated employee productivity bottlenecks
\end{itemize}}

\cventry{2017-2018}{Mathematics Tutor}{West Valley College}{Saratoga}{}{Tutored disadvantaged students in algebra, statistics, and fundamental concepts in mathematics.}

\section{Skills}
	\cvitem{Office}{Word, Excel, Powerpoint}
	\cvitem{Computers}{Microsoft, Linux/Unix, Android and iOS}
	\cvitem{Programming}{Web development, data analysis}
	\cvitem{Databases}{Lexis}

\begin{hidden}
\end{hidden}

\clearpage
%-----       letter       ---------------------------------------------------------
% recipient data
\recipient{Joseph D. Kostmayer}{Law Offices of Joseph D. Kostmayer\\95 S Market St, Ste 300\\San Jose}
\date{September 27, 2018}
\opening{Dear Mr. Kostmayer,}
\closing{Yours respectfully,}
% \enclosure[Attached]{curriculum vit\ae{}}          % use an optional argument to use a string other than "Enclosure", or redefine \enclname
\makelettertitle

As a paralegal I would bring my skill set in philosophical reasoning, technology, and mathematics as well as my enthusiasm for problem-solving.

\textit{With regards to client acquisition:}
I am knowledgable in the administration of websites. I am capable of creating and maintaining a solid web presence for the firm which would make it it easier for potential clients to find it. I know of various database systems and how they might be used to keep track of which clients are involved with the firm.

\textit{With regards to law office management:}
I can troubleshoot almost any technical issue that might arise; computers like me and I am comfortable with adapting to new technologies. As such, I would be able to help a transition to a new software platform if it arose. I would also be capable of working with billing (which is largely a matter of multiplication and spreadsheet entry).

\textit{With regards to legal argumentation:}
My training in philosophy involved a great deal of definitional nitpicking, work with technical subjects, close reading of texts and, of course, arguing for positions. In combination with mathematics I have a strong grasp of logic. I would be an effective sounding board for brainstorming in discussions. 


\makeletterclosing
\end{document}
%% end of file `template.tex'.
